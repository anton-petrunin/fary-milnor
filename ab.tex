\section{Alexander--Bishop}

Here we sketch the proof given by Stephanie Alexander and Richard Bishop \cite{alexander-bishop}.
This proof was designed to work for more general ambient spaces.
As a result, it is more elementary.

In the proof, we construct a total-curvature-decreasing deformation of a given knot into a doubly covered
bigon.
The statement follows since the latter has total curvature $4\cdot\pi$.

\parit{Proof of the main theorem.}
Let $\alpha=p_1\dots p_n$ be a nontrivial knot;
that is, one can not get a triangle from $\alpha$ by applying a sequence of triangular isotopies defined in the introduction.

If $n=3$ the polygonal curve $\alpha$ is a triangle.
Therefore, by definition, $\alpha$ is a trivial knot --- there is nothing to show.

Consider the smallest $n$ for which the statement fails;
that is, there is a nontrivial knot $\alpha\z=p_1\dots p_n$ such that
\[\tc\alpha<4\cdot\pi.
\eqlbl{eq:<4pi}\]
We use the indexes modulo $n$; that is, $p_0=p_n$, $p_1=p_{n+1}$ and so on.
Without loss of generality, we may assume that $\alpha$ is in \emph{general position}; 
this time it means that no four vertexes of $\alpha$ lie on one plane. 

Set $\alpha_0=\alpha$.
If the solid triangle $\solidtriangle p_{0}p_1p_{2}$ intersects $\alpha_0$ only in the two adjacent edges,
then applying the corresponding triangular isotopy, we get a knot $\alpha'_0$ with $n-1$ edges that is inscribed in $\alpha_0$. Therefore
\[\tc\alpha_0\ge \tc\alpha_0'.\]
On the other hand, by the induction hypothesis 
\[\tc{\alpha_0'}\ge 4\cdot\pi,\]
which contradicts \ref{eq:<4pi}.

\begin{wrapfigure}{r}{35 mm}
\vskip-0mm
\centering
\includegraphics{mppics/pic-17}
\vskip0mm
\end{wrapfigure}

Let $w'_1$ be the first point on the edge $[p_1,p_2]$ such that the line segment $[p_0,w'_1]$ 
intersects $\alpha_0$.
Denote a point of intersection by $y_1$.

Choose a point $w_1$ on $[p_1,p_2]$ a bit before $w'_1$.
Denote by $x_1$ the point on $[p_0,w_1]$ that minimizes the distance to $y_1$.
This way we get a closed polygonal curve 
$\alpha_1\z=w_1p_2\dots p_n$ with two marked points $x_1$ and $y_1$.
Denote by $m_1$ the number of edges in the arc $x_1w_1\dots y_1$ of $\alpha_1$.

Note that $\alpha_1$ is isotopic to $\alpha_0$;
in particular, $\alpha_1$ is a nontrivial knot.

Now let us repeat the procedure for the adjacent edges $[w_1,p_2]$ and $[p_2,p_3]$ of $\alpha_1$.
If the solid triangle $\solidtriangle w_1p_2p_3$ intersects $\alpha_1$ only at these two adjacent edges, then we get a contradiction the same way as before.
Otherwise, we get a new knot $\alpha_2=w_1w_2p_3\dots p_n$ with two marked points $x_2$ and $y_2$.
Denote by $m_2$ the number of edges in the polygonal curve $x_2w_2\dots y_2$.

Note that the points $x_1,x_2,y_1,y_2$ cannot appear on $\alpha_2$ in the same cyclic order;
otherwise the polygonal curve $x_1x_2y_1y_2$ can be made to be arbitrarily close to a doubly covered bigon which again contradicts~\ref{eq:<4pi}.

Therefore we can assume that the arc $x_2w_2\dots y_2$ lies inside the arc $x_1w_1\dots y_1$ in $\alpha_2$
and therefore $m_1>m_2$.

Continuing this procedure we get a sequence of polygonal curves $\alpha_i\z=w_1\dots w_i p_{i+1}\dots p_n$ with marked points $x_i$ and $y_i$ such that the number of edges $m_i$ from $x_i$ to $y_i$ decreases as $i$ increases.
Clearly $m_i>1$ for any $i$ and $m_1<n$.
Therefore it requires less than $n$ steps to arrive at a contradiction.
\qeds
