 \section{Minimal-surface proof}

In this section we sketch a proof of the theorem via minimal surfaces.
We start with a polygonal line $\alpha$ with total curvature less than $4\cdot\pi$;
further, show that area-minimizing disc spanned by $\alpha$ has no self-intersections, and therefore $\alpha$ has to be a trivial knot.
So in a way the equation for area-minimizing surface solves our problem, we only need to understand this equation.

The main hiero in this proof is the so-called \emph{monotonicity theorem}.
We will also apply the Doublas--Rado theorem on the existence of area-minimizing disc and reuse the inequality between total curvature and angular length from the F\'ary's proof.

A map from domain of $\mathbb{R}^2$ to $\mathbb{R}^3$ will be called a \emph{surface};
it might have self-intersections and singularities, but we assume it is reasonable, say locally Lipschitz; so we could talk about its area.
A point on the surface might refer to a point in $\mathbb{R}^3$, or the corresponding point in the domain of parameters in $\mathbb{R}^2$;
it should be easy to guess from the context.

We denote by $\mathbb{D}$ the closed disc in the plane.
A surface defined by a map $f\:\mathbb{D}\to\mathbb{R}^3$ will be called a \emph{disc}.
The restriction $f|_{\partial \mathbb{D}}$ will be called boundary line of the disc.

A disc $\Sigma$ is called area-minimizing if it has smallest area among the maps with the given boundary line.
The following statement about area-minimizing discs easy to believe, but not easy to prove. %???REFS

\begin{thm}{Doublas--Rado theorem}\label{thm:min-exists}
Given a simple closed polygonal line $\alpha$ in $\mathbb{R}^3$, there is a area-minimizing disc $\Sigma$ with boundary line $\alpha$; it is a smooth surface, possibly with self-intersections and isolated singularities.

Moreover, if $\Sigma$ has no self-intersections, then it is an embedded smooth surface with no singularities (in this case $\alpha$ is a trivial knot).
\end{thm}

Choose a disc $\Sigma$ in $\mathbb{R}^3$ with boundary line $\alpha$.
Given a point $p\notin \alpha$, let us consider \emph{collared} $\Sigma$ with respect to $p$;
it is a new surface that will be denoted by $\hat\Sigma_p$ that includes $\Sigma$ and the \emph{collar} formed by all rays that start points of $\alpha$ and go in the direction opposite to $p$.
Note that $\hat\Sigma_p$ admits a natural parametrization by the whole plane.

\begin{thm}{Monotonicity theorem}\label{thm:monotonicity}
Let $\Sigma$ be an area-minimizing disc with boundary line $\alpha$.
Given a point $p\notin \alpha$, consider the function 
\[W_p(r)=\area (\hat\Sigma_p\cap \bar B_r(p)),\]
where $\bar B_r(p)$ denotes the ball of radius $r$ centered at $p$.
Then the function $r\mapsto \frac{W_p(r)}{r^2}$
is nondecreasing.

Moreover, 

\begin{subthm}{thm:monotonicity:a}
$\lim_{r\to\infty}\frac{W_p(r)}{r^2}=\tfrac12\cdot \Psi_p(\alpha)$, where $\Psi_p(\alpha)$ denotes angular length of $\alpha$ with  respect to $p$.
\end{subthm}

\begin{subthm}{thm:monotonicity:b}
If $p\in \Sigma$, then $\lim_{r\to0}\frac{W_p(r)}{r^2}\ge \pi$.
\end{subthm}


\end{thm}

The argument presented below was found by the second author \cite{St_structure};
it produces slightly weaker result than the classical monotonicity formula, but works for a wider class of ambent spaces.

\parit{Proof.}
Denote by $\lambda_p(r)$ the curve of intersection of the sphere $\partial B_r(p)$ with $\hat\Sigma_p$;
set $\ell(r)\z=\length[\lambda_p(r)]$.
Observe that 
\[W_p'(r)\ge \ell(r)\]
for almost all $r$.
(Formally speaking, this inequality follows from the so-called \emph{coarea formula}.)

Set $\Delta_r=\hat\Sigma_p\cap \bar B_r(p)$;
it is a surface bounded by $\lambda_p(r)$.
Let $\Delta_r'$ be a copy of $\Delta_r$ rescaled with center at $p$ and factor $1-\eps$.
Let us reconnect every point $x\in \partial \Delta_r$ to the corresponding point $x'=(1-\eps)\cdot x\in \partial\Delta_r'$ by a line segment $[x,x']$.
The line segments form an annulus $A_r$, together with $\Delta_r'$, it forms a surface that shares its boundary with $\Delta_r$.
Note that $A_r\cup\Delta_r'$ include all points of $\Delta_r$ that lies on the collar of $\Sigma$.
Therefore $\Delta_r'\cup A_r$ differ from $\Delta_r$ only inside $\Sigma$.
Since $\Sigma$ is area-minimizing, we get that 
\[\area \Delta_r'+\area A_r\ge \area \Delta_r\eqlbl{area=<area}\]
for any $r>0$.
Observe that 
\[
\area \Delta_r=W_p(r),
\qquad
\area \Delta_r'=(1-\eps)^2\cdotW_p(r),
\qquad
\area A_r=\eps\cdot(1-\tfrac\eps2)\cdot r\cdot \ell(r).
\]
Applying \ref{area=<area} for $\eps\to0+$, we get
\[r\cdot \ell(r)\ge 2\cdotW_p(r).\]
Therefore, we get
\[r\cdot W_p'(r)\ge 2\cdotW_p(r)\]
for almost all $r$.
If $W_p$ is smooth, then this inequality implies the main statement.
Since $W_p$ is only nondecreasing, the same can be proved in general case by applying Lebesgue theorem.

The last statement is evident for smooth points of $\Sigma$.
Since smooth points are dense everywhere in $\Sigma$, and $p\mapsto W_p(r)$ is a continuous function,
the main part of theorem implies that $W_p(r)\ge\pi\cdot r^2$ for \emph{any} point $p\in\Sigma$.
Hence the second statement follows.

To prove \ref{SHORT.thm:monotonicity:a}, observe that up to a fixed error $W_p(r)$ is the area of the ball of radius $r$ in the cone over $\alpha$ with the tip at $p$.
It follows that $W_p(r)/r^2$ approaches the area of unit ball in this cone --- hence the result.
\qeds

\parit{Minimal-surface proof.}
Suppose that $\tc\alpha<4\cdot\pi$.
Consider the area-minimizing surface $\Sigma$ with the boundary line $\alpha$; it exists by \ref{thm:min-exists}.
If $\Sigma$ has no self-intersections, then $\alpha$ is a trivial knot.

Suppose $\Sigma$ has a self-intersection at a point $p$.
In this case, the intersection $B_r(p)\cap \Sigma$ is covered by two or more small area-minimizing subdiscs of $\Sigma$.
By \ref{thm:monotonicity:b}, we get 
\[\lim_{r\to0}\frac{W_p(r)}{r^2}\ge 2\cdot\pi.\]

Applying the main statement in the monotonicity theorem we get $\frac{W_p(r)}{r^2}\ge 4\cdot\pi$ for any $r>0$.
By \ref{thm:monotonicity:b} and Proposition~\ref{prop:angular-length} in F\'ary's proof, we get
\[\tc{\alpha}\ge \Psi_p(\alpha)\ge 2\cdot \frac{W_p(r)}{r^2}\ge 4\cdot\pi\]
--- a contradiction.
\qeds


