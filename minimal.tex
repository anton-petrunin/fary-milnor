 \section{Minimal-surface proof}

In this section we sketch a proof of the theorem via minimal surfaces.
We start with a polygonal line with total curvature less than $4\cdot\pi$, show that area-minimizing disc spanned by this line has no self-intersections and therefore it has to be a trivial knot.
So in a way the equation for area minimizing surface solves our problem, we only need to understand this equation.

The main hiero in this proof is the so-called \emph{monotonicity theorem}.
We will also apply the Doublas--Rado theorem on the existence of area-minimizing disc and reuse the inequality between total curvature and angular length from the F\'ary's proof.

A map from domain in the plane to $\mathbb{R}^3$ will be called a \emph{surface};
it might have self-intersections and singularities, but we assume it is reasonable, say locally Lipschitz; so we could talk about its area.
A point on the surface might refer to a point in $\mathbb{R}^3$, or the corresponding point in the domain of parameters;
it should be easy to guess from the context.

We denote by $\mathbb{D}$ the closed disc in the plane.
A surface defined by a map $f\:\mathbb{D}\to\mathbb{R}^3$ will be called a \emph{disc}.
The restriction $f|_{\partial \mathbb{D}}$ will be called boundary line of the disc.

A disc $\Sigma$ is called area-minimizing if it has smallest area among the maps with the given boundary line.
The following statement about area-minimizing discs easy to believe, but not easy to prove. %???REFS

\begin{thm}{Doublas--Rado theorem}\label{thm:min-exists}
Given a simple closed polygonal line $\alpha$ in $\mathbb{R}^3$, there is a area minimizing disc $\Sigma$ with boundary line $\alpha$; it is a smooth surface, possibly with self-intersections and isolated singularities.

Moreover, if $\Sigma$ has no self-intersections, then it is an embedded smooth surface with no singularities (in this case $\alpha$ is a trivial knot).
\end{thm}

Choose a disc $\Sigma$ in $\mathbb{R}^3$ with boundary line $\partial\Sigma$.
Given a point $p\notin \partial\Sigma$, let us consider a new surface $\hat\Sigma_p$ --- the \emph{collared} $\Sigma$ with respect to $p$;
the surface $\hat\Sigma_p$ includes $\Sigma$ and the \emph{collar} formed by all rays that start points of $\partial\Sigma$ and go in the direction opposite to $p$.
Note that $\hat\Sigma_p$ admits a natural parametrization by the whole plane.

\begin{thm}{Monotonicity theorem}
Let $\Sigma$ be an area-minimizing disc with boundary line $\partial \Sigma$.
Given a point $p\notin \partial\Sigma$, consider the function 
\[\Theta_p(r)=\area (\hat\Sigma_p\cap \bar B_r(p)),\]
where $\bar B_r(p)$ denotes the ball of radius $r$ centered at $p$.
Then the function $r\mapsto \frac{\Theta_p(r)}{r^2}$
is nondecreasing.

Moreover, if $p\in \Sigma$, then $\lim_{r\to0}\frac{\Theta_p(r)}{r^2}\ge \pi$.
\end{thm}

The argument presented below was found by the second author \cite{St_structure};
it is not as general as the classical monotonicity formula, but more elementary and therefore more suitable for generalizations.

\parit{Proof.}
Denote by $\lambda_p(r)$ the curve of intersection of the sphere $\partial B_r(p)$ with $\hat\Sigma_p$;
set $\ell(r)\z=\length[\lambda_p(r)]$.
Observe that 
\[\Theta_p'(r)\ge \ell(r)\]
for any $r$ where the left-hand side is defined.
This inequality follows from the so-called \emph{coarea formula}, but it should be obvious anyway.

Let $\Delta_r$ be the connected component of $p$ in $\hat\Sigma_p\cap \bar B_r(p)$;
it is a surface bounded by $\lambda_p(r)$.
Let $\Delta_r'$ be a copy of $\Delta_r$ rescaled with center at $p$ and factor $1-\eps$.
Let us reconnect every point $x\in \partial \Delta_r$ to the corresponding point $x'=(1-\eps)\cdot x\in \partial\Delta_r'$ by a line segment $[x,x']$.
The line segments form an annulus $A_r$, together with $\Delta_r'$, it forms a surface that shares its boundary with $\Delta_r$.
Note that $A_r$ and $\Delta_r'$ include all points of $\Delta_r$ that lies on the collar of $\Sigma$.
So these two surfaces $\Delta_r'\cup A_r$ and $\Delta_r$ differ only in $\Sigma$.
Since $\Sigma$ is area-minimizing, we get that 
\[\area \Delta_r'+\area A_r\ge \area \Delta_r\eqlbl{area=<area}\]
for any $r>0$.
Observe that 
\[
\area \Delta_r=\Theta_p(r),
\qquad
\area \Delta_r'=(1-\eps)^2\cdot\Theta_p(r),
\qquad
\area A_r=\eps\cdot(1-\tfrac\eps2)\cdot r\cdot \ell(r).
\]
Applying \ref{area=<area} for $\eps\to0+$, we get
\[r\cdot \ell(r)\ge 2\cdot\Theta_p(r).\]
Therefore 
\[r\cdot \Theta_p'(r)\ge 2\cdot\Theta_p(r)\]
if the left-hand side is defined.
If $\Theta_p$ is smooth, then it implies the statement.
In general $\Theta_p$ is only nondecreasing, so its derivative is defined almost everywhere;
it turns to be sufficient after a bit of  acrobatic in analysis.

The last statement is evident for smooth points of $\Sigma$.
Since smooth points are dense everywhere in $\Sigma$, and $p\mapsto \Theta_p(r)$ is a continuous function,
the main part of theorem implies that $\Theta_p(r)\ge\pi\cdot r^2$ for \emph{any} point $p\in\Sigma$.
Hence the second statement follows.
\qeds

\parit{Minimal-surface proof.}
Consider the area-minimizing surface $\Sigma$ with the boundary line $\alpha$; it exists by \ref{thm:min-exists}.
Suppose $\Sigma$ has a self-intersection at a point $p$.
In this case the intersection $B_r(p)\cap \Sigma$ is covered by two or more small area minimizing subdiscs of $\Sigma$.
By the second statement in the monotonicity theorem, we get $\Theta_p(r)\ge 4\cdot\pi$ for all small $r$.

Applying the main statement in the monotonicity theorem we get $\Theta_p(r)\ge 4\cdot\pi$ for any $r>0$.
Note that $\Theta_p(r)\to\tfrac12\cdot\Psi_p(\alpha)$ as $r\to \infty$,
where $\Psi_p(\alpha)$ denotes the angular length of $\alpha$ with respect to $p$.
By Proposition~\ref{prop:angular-length} in F\'ary's proof, we get
\[\tc{\alpha}\ge \Psi_p(\alpha)\ge 2\cdot \Theta_p(r)\ge 4\cdot\pi.\]
\qedsf


