In addition to the proofs of F\'ary and Milnor,
we will discuss a proof based on the existence \emph{alternated quadrisecant}, a construction of Erika Pannwitz \cite{pannwitz}.
This proof might be the earliest; according to a note at the end of Fáry’s article, it was found (but not published) by Heinz Hopf.
We will also discuss the proof Stephanie Alexander and Richard Bishop --- this is the most elementary (but not the simplest) proof, as a result it is most suitable for generalizations.
Finally we discuss the \emph{minimal-surface} proof;
it is based on a theorem Tobias Ekholm, Brian White, and Daniel Wienholtz \cite{EWW_embed} which was a breakthrough in minimal surface theory at the time.
































\subsection*{Milnor's proof}

\parit{Milnor's proof.}
Let $\alpha$ be a simple closed polygonal curve.
Assume $\tc{\alpha}<4\cdot\pi$.
Then by the Crofton-type formula (\ref{prop:tc-crofton}), we get that
\[\tc{\alpha_u}<4\cdot\pi
\eqlbl{eq:<4pi}\]
for some unit vector $u$.
Moreover, we can assume that $u$ points in a generic direction;
that is, $u$ is not perpendicular to any edge or diagonal of $\alpha$.

Since the curve $\alpha_u$ runs back and forth along one line, 
every turn contributes $\pi$ to  $\tc{\alpha_u}$.
Therefore $\tc{\alpha_u}=n\cdot\pi$, where $n$ is the number of changes of direction.
Since $\alpha_u$ is closed, $n$ has to be even, so $\tc{\alpha_u}$ may take values $2\cdot\pi$, $4\cdot\pi$, $6\cdot\pi$ and so on.

By \ref{eq:<4pi}, we have $\tc{\alpha_u}= 2\cdot\pi$;
that is, $\alpha_u$ has exactly $2$ turns.
If we rotate the space so that $u$ points upward,
then the height function has exactly one minimum and one maximum;
by Proposition~\ref{prop:one-max-one-min}, $\alpha$ is a trivial knot --- hence the result.
\qeds

























\begin{wrapfigure}{r}{30 mm}
\vskip-0mm
\centering
\includegraphics{mppics/pic-20}
\vskip0mm
\end{wrapfigure}

\begin{thm}{Exercise}
Suppose that a closed curve $\alpha$ crosses a line at four points $a$, $b$, $c$ and $d$.
Assume that the points $a$, $b$, $c$ and $d$ appear on the line in that order and they appear on the curve $\alpha$ in the order $a$, $c$, $b$, $d$.
Show that 
\[\tc\alpha\ge 4\cdot\pi.\]
\end{thm}

A line crossing a knot at four points as in the exercise is called \emph{alternating quadrisecants}.
It turns out that any nontrivial knot admits an alternating quadrisecants \cite{denne};
it provides yet another proof of the F\'ary--Milnor theorem.


\begin{thm}{Advanced exercise}
Show that given any real number $\Phi$ there is a knot $\beta$ such that any knot isotopic to $\beta$ has total curvature at least~$\Phi$.   
\end{thm}

\parit{Hint:} Use that there are knots with arbitrary large \emph{bridge number}, see for example \cite{schultens} and the references therein.
