\section{Tame knots}

Before diving into to the proofs, we need to remind the definition of \emph{knot} and \emph{nontrival knot}.
Despite that intuitively knot as a knotted rope,
the simplest formal definition use polygonal lines.
It turns out that this definition is also suits best for our purposes.

A \emph{knot} (more precicely \emph{tame knot}) is a simple closed polygonal line in the Euclidean space~$\mathbb{R}^3$ (simple means no self-intersections).

The notation $\triangle abc$ is used for the triangle $abc$; that is, a polygonal line with three edges and vertexes $a$, $b$ and $c$.
Let us denote by $\solidtriangle abc$ the convex hull of the points $a$, $b$ and $c$; $\solidtriangle abc$ is the solid triangle with the vertexes $a$, $b$ and $c$.
The points $a$, $b$ and $c$ are assumed to be distinct, but they might lie on one line;
that is, for us a degenerate triangle counts as a legitimate triangle.

We define a \emph{triangular isotopy of a knot} to be the generation of a new knot from the original one by means of the
following two operations:

Assume $[pq]$ is an edge of the knot and $x$
is a point such that the solid triangle $\solidtriangle pqx$  has no common points with the knot except for the edge $[pq]$.
Then we can replace the edge $[pq]$ in the knot by the two adjacent edges $[px]$ and $[xq]$.

We can also perform the inverse operation.
That is, if for two adjacent edges $[px]$ and $[xq]$ of a knot the triangle
$\solidtriangle pqx$ has no common points with the knot except for the points on the edges $[px]$ and $[xq]$,
then we can replace the two adjacent edges $[px]$ and $[xq]$ by the edge $[pq]$.

Polygons that arise from one another by a finite sequence of
triangular isotopies are called \emph{isotopic}.

\begin{wrapfigure}{r}{22 mm}
\vskip-4mm
\centering
\includegraphics{mppics/pic-18}
\vskip0mm
\end{wrapfigure}

A knot that is not isotopic to a triangle is called nontrivial.

The trefoil knot shown on the diagram gives a simple example of nontrivial knot.
A proof that the trefoil knot is nontrivial can be found in any textbook on knot theory.
The most elementary and visual proof is based on the so called \emph{tricolorability} of knot diagrams.   

\begin{thm}{Exercise}\label{ex:triangle-isotopy}
Let $x$ and $y$ be two points on the adjacent edges $[p_1p_2]$ and $[p_2p_3]$ of a knot $\beta=p_1p_2p_3\dots p_n$.
Assume that the solid triangle $\solidtriangle xp_2y$ intersects $\beta$ only along $[xp_2]\cup [p_2y]$.
Show that the knot $\beta'\z=p_1xyp_3\dots p_n$ is isotopic to $\beta$.
\end{thm}

The total curvature of polygonal line is defined as the sum of its external angles.
It is well known that total curvature of a curve cannot be smaller than total curvature of an insribed polygonal line.
In fact the total curvature of closed curve can be defined as the least upper bound on the total curvature of inscrived polygonal lines.
This definition agrees for the smooth curves and it makes sense for any simple curve.

Therefore the theorem has the following reformulation which is stronger than the original theorem.
This is the theorem we will discuss further.

\begin{thm}{Reformulation}\label{thm:fary-milnor}
The total curvature of any nontrival knot is bigger than $4\cdot\pi$. 
\end{thm}


