\section*{Introduction}

The following problem was posted by Karol Borsuk \cite{borsuk}.

\smallskip

\textit{Show that the total curvature of any nontrivial knot is at least $4\cdot\pi$.}

\smallskip

It is known for many proofs based on different ideas.
We sketch several solutions, one solution per section;
each can be read independently.

This problem also has a number of refinements and generalizations;
in particular, a strict inequality holds --- this is the famous \emph{F\'ary--Milnor theorem}.
However, for the sake of simplicity, we stick to the original formulation.

In order to continue, we need to agree on a definition of knot and explain what a \emph{nontrivial knot} is.
Despite the intuitive idea of a knot as a \emph{cyclic rope},
the simplest formal definition uses polygonal curves.
It turns out that this definition is also best suited for our purposes.

A \emph{knot} (more precisely \emph{tame knot}) is a simple closed polygonal curve in the Euclidean space~$\mathbb{R}^3$ (\emph{simple} means \emph{no self-intersections}).

The solid triangle with vertices $a$, $b$, and $c$ will be denoted by $\solidtriangle abc$.
It is defined as the convex hull of the points $a$, $b$, and $c$;
the points $a$, $b$ and $c$ are assumed to be distinct, but they might lie on one line.

We define a \emph{triangular isotopy} of a knot to be the generation of a new knot from the original one by means of the
following two operations:

Assume $[p,q]$ is an edge of the knot and $x$
is a point such that the solid triangle $\solidtriangle pqx$  has no common points with the knot except for the edge $[p,q]$.
Then we can replace the edge $[p,q]$ with the two adjacent edges $[p,x]$ and $[x,q]$.

We can also perform the inverse operation.
That is, if for two adjacent edges $[p,x]$ and $[x,q]$ of a knot the triangle
$\solidtriangle pqx$ has no common points with the knot except for the points on the edges $[p,x]$ and $[x,q]$,
then we can replace $[p,x]$ and $[x,q]$ by one edge $[p,q]$.

Polygons that arise from one another by a finite sequence of
triangular isotopies are called \emph{isotopic}.
A knot that is not isotopic to a triangle (that is, a simple polygonal curve with three vertices) is called \emph{nontrivial}.

\begin{wrapfigure}{r}{22 mm}
\vskip-3mm
\centering
\includegraphics{mppics/pic-18}
\vskip0mm
\end{wrapfigure}

The trefoil knot shown on the diagram to the right gives a simple example of a nontrivial knot.
A proof that it is nontrivial can be found in any textbook on knot theory.
The most elementary and visual proof is based on the so-called \emph{tricolorability} of knot diagrams \cite[Section 1.5]{adams}.

The total curvature of a smooth curve is usually defined as the integral of its curvature.
For polygons, it is defined as the sum of its external angles.
It is well known that the total curvature of a curve cannot be smaller than the total curvature of an inscribed polygonal curve\amm{ (see, for example, \cite{petrunin-zamora})}{}.
In fact, the total curvature of a curve can be defined as the least upper bound on the total curvature of inscribed polygonal curves \cite{aleksandrov-reshetnyak, sullivan-curves}.
This definition agrees with the definition given for smooth curves, and it makes sense for any simple curve.
The total curvature of a curve $\alpha$ will be denoted by $\Phi(\alpha)$.

It follows that the problem allows the following reformulation which we are going to prove.

\parbf{Main theorem.}
\textit{$\Phi(\alpha)\ge 4\cdot\pi$ for any nontrivial knot $\alpha$.}


