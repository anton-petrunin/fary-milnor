\section{Introduction}

First let us recall the formulation of F\'ary--Milnor theorem:

\begin{thm}{Theorem}\label{thm:fary-milnor}
The total curvature of any nontrival knot is bigger than $4\cdot\pi$. 
\end{thm}


This theorem is known for many many proofs based on different (beautiful) ideas which will be the main subject of this note.

The question was raised by Karol Borsuk \cite{borsuk} and answered independently by Istv\'an F\'ary and John Milnor \cite{fary, milnor}.
Latter other proofs were found.
Among these we will discuss a proof based on alternated quadrisecant, a construction of Erika Pannwitz \cite{pannwitz}.
This proof might be the earliest; according to a note at the end of Fáry’s article, it was found (but not published) by Heinz Hopf.
Another proof we will discuss is due to Stephanie Alexander and Richard Bishop --- the most elementary proof and therefore sufficiently flexible for generalizations.
Finally we discuss the minimal-surface proof;
it is based on a theorem Tobias Ekholm, Brian White, and Daniel Wienholtz \cite{EWW_embed} which was a breakthrough in minimal surface theory at the time.
