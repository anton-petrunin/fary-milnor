\section{Introduction}

Here is our main hiero:

\begin{thm}{Theorem}\label{thm:fary-milnor}
The total curvature of any nontrival knot is at least $4\cdot\pi$. 
\end{thm}

This theorem is known for many many proofs based on different (and beautiful) ideas which is the subject of this note.

The question was raised by Karol Borsuk \cite{borsuk} and answered independently by Istv\'an F\'ary and John Milnor \cite{fary, milnor}.
In fact, a strict inequality was proved, but for the sake of simplify we stick to the original question of Borsuk.

In addition to the proofs of F\'ary and Milnor,
we will discuss a proof based on the existence \emph{alternated quadrisecant}, a construction of Erika Pannwitz \cite{pannwitz}.
This proof might be the earliest; according to a note at the end of Fáry’s article, it was found (but not published) by Heinz Hopf.
We will also discuss the proof Stephanie Alexander and Richard Bishop --- this is the most elementary (but not the simplest) proof, as a result it is most suitable for generalizations.
Finally we discuss the \emph{minimal-surface} proof;
it is based on a theorem Tobias Ekholm, Brian White, and Daniel Wienholtz \cite{EWW_embed} which was a breakthrough in minimal surface theory at the time.

\parbf{Definitions.}
In order to continue we need to stick to a definition knot and say what \emph{nontrival knot} is.
Despite that intuitively knot as a \emph{cyclic rope},
the simplest formal definition use polygonal lines.
It turns out that this definition is also suits best for our purposes.

A \emph{knot} (more precicely \emph{tame knot}) is a simple closed polygonal line in the Euclidean space~$\mathbb{R}^3$ (simple means no self-intersections).

The notation $\triangle abc$ is used for the \emph{triangle} $abc$; that is, a polygonal line with three edges and vertexes $a$, $b$ and $c$.
Let us denote by $\solidtriangle abc$ the convex hull of the points $a$, $b$ and $c$; $\solidtriangle abc$ is the \emph{solid triangle} with the vertexes $a$, $b$ and $c$.
The points $a$, $b$ and $c$ are assumed to be distinct, but they might lie on one line;
that is, for us a degenerate triangle counts as a legitimate triangle.

We define a \emph{triangular isotopy} of a knot to be the generation of a new knot from the original one by means of the
following two operations:

Assume $[pq]$ is an edge of the knot and $x$
is a point such that the solid triangle $\solidtriangle pqx$  has no common points with the knot except for the edge $[pq]$.
Then we can replace the edge $[pq]$ in the knot by the two adjacent edges $[px]$ and $[xq]$.

We can also perform the inverse operation.
That is, if for two adjacent edges $[px]$ and $[xq]$ of a knot the triangle
$\solidtriangle pqx$ has no common points with the knot except for the points on the edges $[px]$ and $[xq]$,
then we can replace the two adjacent edges $[px]$ and $[xq]$ by the edge $[pq]$.

Polygons that arise from one another by a finite sequence of
triangular isotopies are called \emph{isotopic}.
A knot that is not isotopic to a triangle is called \emph{nontrivial}.

\begin{wrapfigure}{r}{22 mm}
\vskip-4mm
\centering
\includegraphics{mppics/pic-18}
\vskip0mm
\end{wrapfigure}

The trefoil knot shown on the diagram gives a simple example of nontrivial knot.
A proof that it is nontrivial can be found in any textbook on knot theory.
The most elementary and visual proof is based on the so called \emph{tricolorability} of knot diagrams.

The following exercise should help to feel more comfortable with to the definition.
\textit{Let $x$ and $y$ be two points on the adjacent edges $[p_1p_2]$ and $[p_2p_3]$ of a knot $p_1p_2p_3\dots p_n$.
Assume that the solid triangle $\solidtriangle xp_2y$ intersects $\alpha$ only along $[xp_2]\cup [p_2y]$.
Show that the knot $p_1xyp_3\dots p_n$ is isotopic to $\alpha$.}

\parbf{Reformulation.}
The total curvature of a smooth curve usually defined as the integral of its curvature.
For a polygons, it is defined as the sum of its external angles.
It is well known that total curvature of a curve cannot be smaller than total curvature of an insribed polygonal line.
In fact the total curvature of a curve can be defined as the least upper bound on the total curvature of inscrived polygonal lines.
This definition agrees for the smooth curves and it makes sense for any simple curve.
The total curvature of a curve $\alpha$ will be denoted by $\Phi(\alpha)$.

Therefore the F\'ary--Milnor theorem follows from the next reformulation which we are going to prove.

\begin{thm}{Reformulation}\label{thm:fary-milnor}
For any nontrival knot $\alpha$, we have $\Phi(\alpha)\ge 4\cdot\pi$. 
\end{thm} 


