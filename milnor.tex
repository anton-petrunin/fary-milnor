\section{Milnor's proof}

In the proof we will use the following fact. 

\begin{thm}{Proposition}\label{prop:one-max-one-min}
Assume that a height function $(x,y,z)\to z$ 
has only one local minimum and one local maximum on a closed simple polygonal line and all the vertexes of the polygonal line are at different height.
Then the line is a trivial knot.
\end{thm}

The proof is a simple application of the definition of isotopy, given in the previous section. 

\begin{wrapfigure}{r}{20 mm}
\vskip-0mm
\centering
\includegraphics{mppics/pic-19}
\vskip0mm
\end{wrapfigure}

\parit{Proof.}
Let $\beta=p_1\dots p_n$ be the closed simple polygonal line
such that the height function $(x,y,z)\to z$ has one local minimum one local maximum.
Note that on each of the two arcs of $\beta$ from the min-vertex to the max-vertex the height function increases monotonically.

Consider the three vertexes with the largest height;
they have to include the max-vertex and two more.
Note that these three vertexes are consequent in the polygonal line; 
without loss of generality we can assume that they are $p_{n-1},p_n,p_1$.

Note that the solid triangle $\solidtriangle p_{n-1}p_np_1$ does not intersect any edge $\beta$ except the two adjacent edges $[p_{n-1}p_n]\cup[p_np_1]$.
Indeed, if $\solidtriangle p_{n-1}p_np_1$ intersects $[p_1p_2]$,
then, 
since $p_2$ lies below $\solidtriangle p_{n-1}p_np_1$,
$[p_1p_2]$ must intersect $[p_{n-1}p_n]$
which is impossible since $\beta$ is simple.
The same way one can show that $\solidtriangle p_{n-1}p_np_1$ can not intersect $[p_{n-2}p_{n-1}]$.
The remaining edges lie below $\solidtriangle p_{n-1}p_np_1$, hence they can not intersect this triangle.

Applying a triangular isotopy, to $\solidtriangle p_{n-1}p_np_1$ we get a closed simple polygonal line $\beta'\z=p_1\dots p_{n-1}$ which is isotopic to $\beta$.

Since all the vertexes $p_i$ have different height,
the assumption of the proposition holds for $\beta'$.

Repeating this procedure $n-3$ times we get a triangle.
Hence $\beta$ is a trivial knot.
\qeds

\parit{Milnor's proof of \ref{thm:fary-milnor}.}
Let $\alpha$ be a simple closed polygonal line.
Assume its total curvature is less that $4\cdot\pi$.
Then by Proposition~\ref{prop:tc-crofton}, 
\[\tc\alpha_u<4\cdot\pi\]
for some unit vector $u$.
Moreover, we can assume that $u$ points in a generic direction;
that is, $u$ is not perpendicular to any edge or diagonal of $\alpha$.

The total curvature of $\alpha_u$ is $\pi$ times the number of turns of $\alpha_u$
which has to be an even number.
It follows that the number of turns of $\alpha_u$ is at most $2$;
it cannot be less than 2 for a generic direction, therefore it is exactly $2$.

That is, if we rotate the space so that $u$ points upward,
then the height function has exactly one minimum and one maximum;
by Proposition~\ref{prop:one-max-one-min}, $\alpha$ is a trivial knot --- hence the result.
\qeds
