\section{Milnor--Fenner}

One of the first solutions to the problem was found by John Milnor \cite{milnor}.
In this section, we present an amusing interpretation of his proof found by Stephen Fenner \cite{ferner}.
Just like the original version, it is based on the following sufficient condition for the triviality of a knot.

\begin{thm}{Proposition}\label{prop:one-max-one-min}
Assume that a height function $(x,y,z)\to z$ 
has only one local maximum on a closed simple polygonal curve $\alpha$ and all the vertices of the polygonal curve are at different heights.
Then $\alpha$ is a trivial knot.
\end{thm}

The proof is a straightforward construction of a triangular isotopy. 

\parit{Proof.}
Let $\alpha=p_1\dots p_n$ be the closed simple polygonal curve
such that the height function $(x,y,z)\to z$ has one local maximum and all vertices have different heights.
Note that the height function also has a unique local minimum,
and $\alpha$ can be divided into two arcs from the min-vertex to the max-vertex with monotonic height function.

Consider the three vertexes with the largest height;
they have to include the max-vertex and two more.
Note that these three vertices are consequent in the polygonal curve; 
without loss of generality, we can assume that they are $p_{n-1}$, $p_n$, and $p_1$.

\begin{wrapfigure}{r}{30 mm}
\vskip-0mm
\centering
\includegraphics{mppics/pic-19}
\vskip0mm
\end{wrapfigure}

Note that the solid triangle $\solidtriangle p_{n-1}p_np_1$ does not intersect any edge $\alpha$ except the two adjacent edges $[p_{n-1},p_n]\cup[p_n,p_1]$.
Indeed, if $\solidtriangle p_{n-1}p_np_1$ intersects $[p_1,p_2]$,
then, 
since $p_2$ lies below $\solidtriangle p_{n-1}p_np_1$,
the edge $[p_1,p_2]$ must intersect $[p_{n-1},p_n]$;
the latter is impossible since $\alpha$ is simple.

The same way, one can show that $\solidtriangle p_{n-1}p_np_1$ can not intersect $[p_{n-2},p_{n-1}]$.
The remaining edges lie below $\solidtriangle p_{n-1}p_np_1$, hence they cannot intersect this triangle.

Applying a triangular isotopy, to $\solidtriangle p_{n-1}p_np_1$ we get a closed simple polygonal curve $\alpha'\z=p_1\dots p_{n-1}$ which is isotopic to~$\alpha$.

Since all the vertices $p_i$ have different heights,
the assumption of the proposition holds for $\alpha'$.

Repeating this procedure $n-3$ times we get a triangle.
Hence $\alpha$ is a trivial knot.
\qeds

\parit{Proof of the main theorem.}
Let $\alpha=p_1\dots p_n$ be a nontrivial polygonal knot.
Denote by $v_i$ the unit vector in the direction of $p_{i+1}-p_i$;
we assume that $p_n=p_0$.
Consider the set $U_i$ formed by all unit vectors $u$ such that $\measuredangle(u,v_i)\ge \tfrac \pi 2$ and $\measuredangle(u,v_{i-1})\le \tfrac \pi 2$.
Note that $u\in U_i$ if and only if the function $x\mapsto \langle u,x\rangle$ has a local maximum at $p_i$ on $\alpha$.

Let us choose $(x,y,z)$-coordinates in the space so that $z$-axis points in the direction of $u$.
Then according to \ref{prop:one-max-one-min}, the function $p\mapsto \langle u,p\rangle$ has at least two local maxima on $\alpha$.
It follows that the sets $U_1,\dots,U_n$ cover each point on the unit sphere $\mathbb{S}^2$ twice.

Recall that $\area \mathbb{S}^2=4\cdot\pi$.
Observe that $\phi_i=\measuredangle(v_{i-1},v_i)$ is the external angle of $\alpha$ at $p_i$.
Note that $U_i$ is a slice of the sphere between two meridians meeting at angle $\phi_i$, therefore $U_i$ occupies a $\tfrac{\phi_i}{2\cdot\pi}$ portion of the whole sphere; so, $\area U_i=2\cdot \phi_i$.
Since the sets $U_1, \dots, U_n$ cover $\mathbb{S}^2$ twice, we get
\[\tc\alpha=\phi_1+\dots+\phi_n=\tfrac12\cdot (\area U_1+\dots+\area U_n)\ge \tfrac22\cdot \area\mathbb{S}^2=4\cdot\pi.\]
\qedsf



