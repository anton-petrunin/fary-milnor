\section{Proofs with Crofton-type formula}

The proofs of F\'ary and Milnor are very similar;
both are based on similar formulas for total curvature.

Given a curve $\alpha$ in $\mathbb{R}^3$ and a unit vector $u$, denote by $\alpha_{u^\perp}$ 
and $\alpha_u$ the projection of $\alpha$ to the plane perpendicular to $u$ and the line parallel to $u$ respectively.
Given a function $f\:\mathbb{S}^2\to\mathbb{R}$, let us denote its average value by $\overline{f(u)}$.

\begin{thm}{Crofton-type formula}\label{prop:tc-crofton}
Let $\alpha$ be a polygonal line in $\mathbb{R}^3$.
Then
\begin{align*}
\tc\alpha
&=\overline{\tc{\alpha_{u^\perp}}}
=\overline{\tc{\alpha_u}}.
\end{align*}
\end{thm}


\parit{Proof.}
Observe that it is sufficient to check the identities for polygonal line made of two edges
and in this case it boils to very straightforward calculations.
\qeds

\subsection*{Milnor's proof}

Milnor's proof is based the identity $\tc\alpha=\overline{\tc{\alpha_u}}$ and the following fact. 

\begin{thm}{Proposition}\label{prop:one-max-one-min}
Assume that a height function $(x,y,z)\to z$ 
has only one local minimum and one local maximum on a closed simple polygonal line and all the vertexes of the polygonal line are at different height.
Then the line is a trivial knot.
\end{thm}

The following proof is a straightforward construction of a triangular isotopy. 

\begin{wrapfigure}{r}{20 mm}
\vskip-0mm
\centering
\includegraphics{mppics/pic-19}
\vskip0mm
\end{wrapfigure}

\parit{Proof.}
Let $\alpha=p_1\dots p_n$ be the closed simple polygonal line
such that the height function $(x,y,z)\to z$ has one local minimum one local maximum.
Note that on each of the two arcs of $\alpha$ from the min-vertex to the max-vertex the height function increases monotonically.

Consider the three vertexes with the largest height;
they have to include the max-vertex and two more.
Note that these three vertexes are consequent in the polygonal line; 
without loss of generality we can assume that they are $p_{n-1},p_n,p_1$.

Note that the solid triangle $\solidtriangle p_{n-1}p_np_1$ does not intersect any edge $\alpha$ except the two adjacent edges $[p_{n-1}p_n]\cup[p_np_1]$.
Indeed, if $\solidtriangle p_{n-1}p_np_1$ intersects $[p_1p_2]$,
then, 
since $p_2$ lies below $\solidtriangle p_{n-1}p_np_1$,
the edge $[p_1p_2]$ must intersect $[p_{n-1}p_n]$;
the latter is impossible since $\alpha$ is simple.

The same way one can show that $\solidtriangle p_{n-1}p_np_1$ can not intersect $[p_{n-2}p_{n-1}]$.
The remaining edges lie below $\solidtriangle p_{n-1}p_np_1$, hence they can not intersect this triangle.

Applying a triangular isotopy, to $\solidtriangle p_{n-1}p_np_1$ we get a closed simple polygonal line $\alpha'\z=p_1\dots p_{n-1}$ which is isotopic to $\alpha$.

Since all the vertexes $p_i$ have different height,
the assumption of the proposition holds for $\alpha'$.

Repeating this procedure $n-3$ times we get a triangle.
Hence $\alpha$ is a trivial knot.
\qeds

\parit{Milnor's proof.}
Let $\alpha$ be a simple closed polygonal line.
Assume $\tc{\alpha}<4\cdot\pi$.
Then by the Crofton-type formula (\ref{prop:tc-crofton}), we get that
\[\tc{\alpha_u}<4\cdot\pi
\eqlbl{eq:<4pi}\]
for some unit vector $u$.
Moreover, we can assume that $u$ points in a generic direction;
that is, $u$ is not perpendicular to any edge or diagonal of $\alpha$.

Since the curve $\alpha_u$ runs back and forth along one line, 
every turn contributes $\pi$ to  $\tc{\alpha_u}$.
Therefore $\tc{\alpha_u}=n\cdot\pi$, where $n$ is the number of changes of direction.
Since $\alpha_u$ is closed, $n$ has to be even, so $\tc{\alpha_u}$ may take values $2\cdot\pi$, $4\cdot\pi$, $6\cdot\pi$ and so on.

By \ref{eq:<4pi}, we have $\tc{\alpha_u}= 2\cdot\pi$;
that is, $\alpha_u$ has exactly $2$ turns.
If we rotate the space so that $u$ points upward,
then the height function has exactly one minimum and one maximum;
by Proposition~\ref{prop:one-max-one-min}, $\alpha$ is a trivial knot --- hence the result.
\qeds


\subsection*{F\'ary's proof}

F\'ary's proof is based the identity $\tc\alpha=\overline{\tc{\alpha_{u^\perp}}}$ and the following fact.

Let $\alpha=p_1\dots p_n$ be a simple closed polygonal line in $\mathbb{R}^3$ and $o\notin\alpha$.
Let us define the \emph{angular length} of $\alpha$ with respect to $o$ as the sum
\[\Psi_o(\alpha)=\measuredangle p_{1} o p_{2}+\dots+\measuredangle p_{n-1} o p_{n}+\measuredangle p_{n} o p_{1}.\]


\begin{thm}{Proposition}\label{prop:angular-length}
For any close simple polygonal line and any $o\notin\alpha$, we have 
\[\Psi_o(\alpha)\le \tc{\alpha}.\]
\end{thm}

\parit{Proof.}
Let $\alpha=p_1\dots p_n$; for each $i$, set 
\begin{align*}
\phi_i&=\pi-\measuredangle p_{i-1}p_ip_{i+1},
&
\psi_i&=\measuredangle p_{i-1} o p_{i},
&
\theta_i&=\measuredangle o p_i p_{i+1}.
\end{align*}
Here we assume that indexes are taken modulo $n$; in particular, $p_{n}=p_0$.


Note that $\phi_i$ is the external angle at $p_i$;
therefore 
\[\tc\alpha= \phi_1+\dots+\phi_n\]

\begin{wrapfigure}{r}{20 mm}
\vskip-0mm
\centering
\includegraphics{mppics/pic-15}
\vskip0mm
\end{wrapfigure}

The directions of $p_i-p_{i-1}$, $o-p_i$, and $p_{i+1}-p_i$ form $p_i$ make angles 
$\psi_i+\theta_{i-1}$, $\theta_i$, and $\phi_i$ to each other.
Applying the triangle inequality for these angles, we get
\[\phi_i\ge \psi_i+\theta_{i-1}-\theta_i.\]
Summing up these inequalities for all $i$, we get
\[\phi_1+\dots+\phi_n\ge \psi_1+\dots+\psi_n,\]
and the result follows.
\qeds


\begin{wrapfigure}{r}{30 mm}
\vskip-0mm
\centering
\includegraphics{mppics/pic-13}
\vskip0mm
\end{wrapfigure}

\parit{Skettch of F\'ary's proof.}
Consider a projection of the knot to a plane in general position.
That is, we assume that the self-intersections of the projection are at most double and the projection of each edge is not degenerate.
The obtained closed polygonal line $\alpha_{u^\perp}=p_1p_2\dots p_n$ divides the plane into domains, one of which is unbounded, denote it by $U$, and the others are bounded.

First note that all domains can be colored in a chessboard order;
that is, they can be colored in black and white in such a way that domains with common borderline get different colors.
If the unbounded domain is colored in white and every other domain is colored in black then one can untie the knot by flipping these domains one by one.
(It is instructive to give a formal proof of the last statement; that is, show that if the only undbounded domain is white then $\alpha$ is isotopic to a triangle.) 


\begin{wrapfigure}{r}{30 mm}
\vskip-4mm
\centering
\includegraphics{mppics/pic-14}
\vskip0mm
\end{wrapfigure}

Therefore among the bounded domains there is a white domain, denote it by $D$.
The domain $D$ cannot adjoin %the word adjoin is used for countries, and I guess it can be used in the plane.
$U$, since they have the same color.
Fix a point $o$ in this domain.

Since any ray from $o$ crosses $\alpha_{u^\perp}$ twice, we get $\Psi_o(\alpha_{u^\perp})\ge 4\cdot\pi$;
that is, the angular length of $\alpha_{u^\perp}$ with respect to $o$ is at least $4\cdot\pi$. 
By \ref{prop:angular-length}, we have 
\[\tc{\alpha_{u^\perp}}\ge4\cdot\pi.\]
This is true for any $u$ in general position.
The remaining directions contribute nothing to the average value.
It remains to apply the Crofton-type formula $\tc{\alpha}=\overline{\tc{\alpha_{u^\perp}}}$.
\qeds



\begin{thm}{Exercise}
Construct a closed smooth simple curve with total curvature arbitrarily close to $2\cdot\pi$ such that its projection to any plane has at least $10$ self-intersections.   
\end{thm}

