 \documentclass[a4paper,12pt]{amsart}
\usepackage{color}
\usepackage{amsmath,amssymb,amscd,amsfonts,amsthm,stmaryrd}
\usepackage[mathscr]{eucal}
\usepackage{xcolor}
%\usepackage{showkeys}
\usepackage{graphicx}

\usepackage{nicefrac}

\usepackage{color}
\newcommand{\blue}[1]{{\color{blue}{#1}}}
\newcommand{\red}[1]{{\color{red}{#1}}}
\newcommand{\green}[1]{{\color{ForestGreen}{#1}}}
\newcommand{\gray}[1]{{\color{lightgray}{#1}}}


\numberwithin{equation}{section}

\def\a{\operatorname{area}}
\def\cone{\operatorname{cone}}
\def\Hess{\operatorname{Hess}}
\def\div{\operatorname{div}}
\def\dvol{\operatorname{dvol}}
\def\Jac{\operatorname{Jac}}
\def\length{\operatorname{length}}
\def\aplim{\operatorname{aplim}}
\def\tr{\operatorname{tr}}
\def\Fill{\operatorname{Fill}}
\def\dev{\operatorname{dev}}
\def\ext{\operatorname{ext}}
\def\diam{\operatorname{diam}}
\def\rank{\operatorname{rank}}
\def\pol{\operatorname{pol}}
\def\rad{\operatorname{rad}}
\def\id{\operatorname{id}}



\def\acts{\curvearrowright}
\def\D{\partial}
\def\restr{\mbox{\large \(|\)\normalsize}}
\def\R{\mathbb R}
\def\Z{\mathbb Z}
\def\S{\mathbb S}
\def\E{\mathbb E}
\def\N{\mathbb N}
\def\Ha{\mathcal H}

\newcommand{\F}{F}
\newcommand{\G}{\Theta}
\newcommand{\Fi}{\F_\infty}
\newcommand{\Gi}{\G_\infty}

\def\al{\alpha}
\def\de{\delta}
\def\del{\delta}
\def\ka{\kappa}
\def\De{\Delta}
\def\Del{\Delta}
\def\eps{\epsilon}
\def\ga{\gamma}
\def\Ga{\Gamma}
\def\half{{1 \over 2}}
\def\la{\lambda}
\def\La{\Lambda}
\def\lra{\longrightarrow}
\def\Lra{\Longrightarrow}
\def\Tau{{\cal T}}
\newcommand{\oa}[1]{\stackrel{\ra}{#1}}
\def\overarrow{\oa}
\def\embed{\hookrightarrow}
\def\ol{\overline}
\def\om{\omega}
\def\Om{\Omega}
\def\ra{\rightarrow}
\def\Ra{\Rightarrow}
\def\si{\sigma}
\def\Si{\Sigma}
\def\tangle{\angle_{Tits}}
\def\tits{\partial_{T}}
\def\geo{\partial_{\infty}}
\def\etits{\partial^{ess}_{T}}




\newcommand{\bH}{{\mathbf H}}
\newcommand{\bI}{{\mathbf I}}
\newcommand{\bZ}{{\mathbf Z}}
\newcommand{\M}{{\mathbf M}}

\newcommand{\cC}{{\mathscr C}}
\newcommand{\cD}{{\mathscr D}}
\newcommand{\cF}{{\mathscr F}}
\newcommand{\cI}{{\mathscr I}}
\newcommand{\cL}{{\mathscr L}}
\newcommand{\cN}{{\mathscr N}}
\newcommand{\cP}{{\mathscr P}}
\newcommand{\cS}{{\mathscr S}}
\newcommand{\cZ}{{\mathscr Z}}
\newcommand{\on}{\:\mbox{\rule{0.1ex}{1.2ex}\rule{1.1ex}{0.1ex}}\:}

\def\wlim{\mathop{\hbox{$\om$-lim}}}

\def\acts{\curvearrowright}
\def\interior{\operatorname{int}}
\def\Isom{\mathop{\hbox{Isom}}}
\def\Stab{\mathop{\hbox{Stab}}}
\def\spt{\operatorname{spt}}

\def\<{\langle}
\def\>{\rangle}

\newcommand{\bb}[1]{\llbracket #1\rrbracket} 
% double brackets, requires package stmaryrd
  

\newcommand{\cs}{\text{\rm c}} %roman subscript for compact support
\newcommand{\C}{\text{\rm C}} %for cones
\newcommand{\loc}{\text{\rm loc}}


\theoremstyle{plain}
\newtheorem{thm}{Theorem}[section]
\newtheorem{lem}[thm]{Lemma}
\newtheorem{prop}[thm]{Proposition}
\newtheorem{cor}[thm]{Corollary}
\newtheorem{slem}[thm]{Sublemma}
\newtheorem{question}[thm]{Question}
\newtheorem{introthm}{Theorem}
\renewcommand{\theintrothm}{\Alph{introthm}}
\newtheorem{introcor}{Corollary}
\renewcommand{\theintrocor}{\Alph{introcor}}

\newtheorem*{conj}{Conjecture}
\newtheorem*{quest}{Question}

\newtheoremstyle{named}{}{}{\itshape}{}{\bfseries}{.}{.5em}{\thmnote{#3} #1}
\theoremstyle{named}
\newtheorem*{namedlemma}{Lemma}

\newtheorem{repropinner}{Proposition}
\newenvironment{reprop}[1]{%
  \renewcommand\therepropinner{#1}%
  \repropinner
}{\endrepropinner}


\theoremstyle{definition}
\newtheorem{dfn}[thm]{Definition}

\theoremstyle{remark}
\newtheorem{rem}[thm]{Remark}
\newtheorem{exmp}[thm]{Example}





\newcommand{\bcl}{\begin{claim}}
\newcommand{\ecl}{\end{claim}}
\newcommand{\bcor}{\begin{cor}}
\newcommand{\ecor}{\end{cor}}
\newcommand{\bdfn}{\begin{dfn}}
\newcommand{\edfn}{\end{dfn}}
%\newcommand{\bdfn}{\begin{definition}}
\newcommand{\ben}{\begin{enumerate}}
\newcommand{\bit}{\begin{itemize}}
\newcommand{\blem}{\begin{lem}}
\newcommand{\bslem}{\begin{slem}}
\newcommand{\bprop}{\begin{prop}}
\newcommand{\bthm}{\begin{thm}}
%\newcommand{\edfn}{\end{definition}}
\newcommand{\een}{\end{enumerate}}
\newcommand{\eit}{\end{itemize}}
\newcommand{\elem}{\end{lem}}
\newcommand{\eslem}{\end{slem}}
\newcommand{\eprop}{\end{prop}}
\newcommand{\ethm}{\end{thm}}

\def\mini{\scriptsize}

\usepackage{hyperref}

%\subjclass[2010]{51E24, 51K10, 53C23, 53C24, 53C35}

%\keywords{Rank Rigidity, Tits boundary, symmetric space, building}


%%%%%%%%%%



\begin{document}

\section{Fary-Milnor after Ekholm-White-Wienholtz}

In this section we want to provide a proof of the Fary-Milnor theorem using minimal surfaces.
The original proof was a breakthrough in minimal surface theory and is due to Ekholm-White-Wienholtz \cite{EWW_embed}. 
Their technique was used in \cite{CG_embed, St_structure} to prove the Fary-Milnor theorem in
non-Euclidean geometries. Our presentation uses arguments from \cite{EWW_embed} and \cite{CG_embed}.
\medskip

Throughout this section, $G(r)=\frac{\log(r)}{2\pi}$ will denote the Green's function on $\R^2$ and  $\rho(x)=\|x-p\|$ will refer to
the distance function to some point $p\in\R^3$.

\blem\label{lem_laplace} 
Let $\Si\subset \R^3$ be an immersed surface. Then except at $p$,
\[\Delta_\Si G(\rho)=\frac{1}{\pi\rho^2}(1-|\nabla_\Si \rho|^2)+\frac{d\rho(H)}{2\pi\rho},\]
where $H$ denotes the mean curvature vector of $\Si$. 
\elem

\proof
We have
\[\Delta_\Si G(\rho)=\tr_\Si\Hess G(\rho)+d G(\rho)(H).\]
Let $\{e_1,e_2\}$ be an orthonormal basis to $\Si$ such that $d\rho(e_1)=|\nabla_\Si \rho|$ and $d\rho(e_2)=0$. 
The claim follows since 
\[\Hess G(\rho)=\frac{1}{2\pi\rho}(\Hess\rho-\frac{1}{\rho}d\rho\otimes d\rho)\] 
and
\[\Hess\rho=\frac{1}{\rho}(\<\cdot,\cdot\>-\<\cdot,\nabla\rho\>^2).\]
\qed

\bdfn
An immersed surface $\Si\subset\R^3$ is called {\em minimal}, provided its mean curvature vanishes everywhere.
\edfn
For a curve $\ga\subset\R^3$ and a point $p\in\R^3\setminus\ga$ we will denote by $C_p(\ga)$ the cone at $p$
over $\ga$:
\[C_p(\ga):=\{p+t(x-p)|\ x\in\ga, 0\leq t\leq 1\}.\]
\medskip

\bcor\label{cor_subhar}
Let $\Si\subset\R^3$ be a minimal surface, and denote by $C=C_p(\D\Si)$ the cone at $p$ over the boundary $\D\Si$.
The $G(\rho)$ is subharmonic on $\Si$ and harmonic on $C\setminus\{p\}$.
\ecor

\proof
Clearly, $|\nabla_\Si\rho|\leq 1$ and $|\nabla_C\rho|\equiv 1$.
Moreover, the mean curvature vector of $\Si$ vanishes while the mean curvature vector of $C$
is orthogonal to $\nabla\rho$. By Lemma~\ref{lem_laplace}, we obtain $\Delta_\Si G(\rho)\geq 0$ and $\Delta_C G(\rho)= 0$
on the complement of $p$. Since $\Si$ is immersed at $p$, $\frac{1}{\rho}\cdot(\Si\cap B_\rho(p))$
converges smoothly to a finite union of unit discs. Hence, $\lim\limits_{\rho\to 0}\frac{|\nabla\rho|-|\nabla_\Si\rho|}{\rho}=0$, and
$G(\rho)$ is subharmonic on all of $\Si$.
\qed


\bdfn
For an immersed surface $\Si\subset\R^3$, we define the {\em density} of $\Si$ at $p$ by
\[\Theta(\Si,p)=\lim\limits_{r\to 0}\frac{\a(\Si\cap B_r(p))}{\pi r^2}.\]
\edfn

The density of an immersed surface $\Si$ counts the number of sheets through a given point.
By the Monotonicity Theorem, the {\em density ratio} $\Theta(\Si,p,r):=\frac{\a(\Si\cap B_r(p))}{\pi r^2}$ of a minimal surface $\Si$
is a non-decreasing function of $r$ as long as $r<d(p,\D\Si)$. The same holds true for surfaces which are conical with respect to $p$.

\bprop[Density comparison]\label{prop_dens_comp}
Let $\Ga\subset\R^3$ be a smooth Jordan curve. Suppose that $\Si\subset\R^3$
is a minimal surface with $\D\Si=\Ga$. Let $p$ be a point in $\Si\setminus\Ga$ and denote by $C$ the
cone from $p$ over $\Ga$. Then
\[\Theta(\Si,p)\leq \Theta(C,p)\]
with equality if and only if $\Si$ lies in a plane. 
\eprop

\proof
For simplicity, we will assume that $C$ is immersed.
Choose a small $\eps>0$ and remove the $\eps$-ball around $p$ from $\Si$ and $C$,
\[\Si_\eps:=\Si\setminus B_\eps(p) \quad\quad C_\eps:=C\setminus B_\eps(p).\]
Denote by $\nu_\Si$ and $\nu_C$ the outward unit normal vectors at the boundaries, tangent to the respective surface.
Then, by Corollary~\ref{cor_subhar} and Cauchy-Schwarz,
\begin{align*}
0\leq&\int_{\Si_\eps}\Delta_\Si G(\rho)\ dA=\int_{\D\Si_\eps}\<\nabla G(\rho),\nu_\Si\>\ ds\\
\leq&\int_{\Si\cap\D B_\eps(p)}\frac{\<\nabla\rho,\nu_\Si\>}{2\pi\eps}\ ds +\int_{\Ga}\frac{1}{2\pi\rho}\ ds.
\end{align*} 
Since $\Si$ is immersed, we have $\<\nabla\rho,\nu_\Si\>\to-1$ uniformly, and therefore
\[\lim\limits_{\eps\to 0}\int_{\Si\cap\D B_\eps(p)}\frac{\<\nabla\rho,\nu_\Si\>}{2\pi\eps}\ ds=
\lim\limits_{\eps\to 0}\frac{\length(\Si\cap\D B_\eps(p))}{2\pi\eps}=\Theta(\Si,p).
\]
Thus,
\[\Theta(\Si,p)\leq \int_{\Ga}\frac{1}{2\pi\rho}\ ds.\]
Applying Corollary~\ref{cor_subhar} again, a similar computation for $C$ shows
\[\Theta(C,p)= \int_{\Ga}\frac{1}{2\pi\rho}\ ds.\]
We leave the equality discussion for the interested reader.
\qed
\medskip

For a point $p\in\R^3$ we will denote by $\pi_p:\R^3\setminus\{p\}\to\D B_1(p)$
the radial projection $\pi_p(x)=p+\frac{x-p}{\|x-p\|}$.

\bprop\label{prop_dens_est}
Let $\Ga\subset\R^3$ be a smooth Jordan curve and let $p$
be a point in $\R^3\setminus\Ga$. Then
\[\length(\pi_p(\Ga))\leq\ka(\Ga).\]
Equivalently,
\[\Theta(C_p(\Ga))\leq\frac{\ka(\Ga)}{2\pi}.\]
\eprop

\proof
After translation and dilatation, we may assume that $p=0$ and that $\Ga$ lies outside the unit ball $B_1(0)$.
Let $S:=C_0(\Ga)\setminus B_1(0)$ be the region of $C_0(\Ga)$ between $\D B_1(0)$ and $\Ga$. By the Gauss-Bonnet theorem,
\[\int_S K\ dA+\int_{\D S}\<k,\nu_S\> ds=2\pi\chi(S), \]
where $k$ is the curvature vector of the curve $\D S$ in $\R^3$, $\nu_S$ is the outward unit normal
of $S$, $K$ is the scalar curvature of $S$, and $\chi(S)$ is the Euler characteristic of $S$.
Since $S$ is a ruled surface homeomorphic to an annulus, we have $K\equiv 0$ and $\chi(S)=0$. Hence
\[0=\int_{\D S}\<k,\nu_S\> ds=\int_{\pi_p(\Ga)}\underbrace{\<k,\nu_S\>}_{\equiv 1} ds+\int_{\Ga}\<k,\nu_S\> ds.\]
The claim follows because the second integral is bounded in absolute value by the total curvature of $\Ga$.
\qed


\bthm\label{thm_embed}
Let $\Ga\subset\R^3$ be a smooth Jordan curve.
Suppose that $\Si\subset\R^3$ is a minimal surface with boundary $\D\Si=\Ga$.
If the total curvature of $\Ga$ satisfies
\[\ka(\Ga)\leq 4\pi,\]
then $\Si$ is embedded. 
\ethm

\proof
To show that $\Si$ is embedded it is enough to show $\Theta(\Si,p)<2$ for all $p\in\Si\setminus\Ga$.
Let $p$ be such a point.
If $\Si$ is totally geodesic, then it is embedded.
Otherwise, by Propositions~\ref{prop_dens_comp} and \ref{prop_dens_est}
\[\Theta(\Si,p)<\Theta(C_p(\Ga))\leq\frac{\ka(\Ga)}{2\pi}\leq 2\]
and the proof is complete.
\qed

\bcor[Fary-Milnor]
Let $\Ga\subset\R^3$ be a smooth Jordan curve of total curvature at most $4\pi$, then $\Ga$ is unknotted.
\ecor

\proof
By the Douglas-Rado theorem there exists a least-area disc $\Si$ filling $\Ga$.
By Theorem~\ref{thm_embed}, $\Si$ is a
smooth embedded disc. If $u:D\to\Si$ denotes a smooth, conformal parametrization of $\Si$ by the
unit disc $D\subset\R^2$, then $h(x,t)=u(tx)$ provides an isotopy $h:\D D\times(0,1]\to\R^3$ between $\Ga$
and the circles $\Ga_t:=u(t\D D)$. Since $u$ has nondegenerated differential at $0$, $\Ga_t$ is unknotted for small enough $t$.
\qed



\section{Fary-Milnor after Schmitz}

Let $c:S^1\to\R^3$ be a polygonal knot with image $\Ga$.
We identify the set of chords of $\Ga$ with the annulus $(\Ga\times\Ga)\setminus\Delta_\Ga$.
We also identify $S^1:=\R/\Z$ and consider the induced map
\[f:\R\times(0,1)\to\Ga\times\Ga;\ (t,h)\mapsto(c(t),c(t+h)).\]
For $(t,h)\in\R\times(0,1)$ we definde the ray 
\[\rho_{(t,h)}(r)=c(t)+r\cdot(c(t+h)-c(t)).\]
In order to estimate the total curvature of $\Ga$, we investigate the set of triple chords. We define 
\[
C_3:=\{(t,h)\in\R\times(0,1)|\ c(t+k)\in\rho_{(t,h)} \text{ for some } k\in(0,1)\}.
\]
Since triple chords come in two types, according to the orientation of $\Ga$, we have the following subsets.
\[
C^-_3:=\{(t,h)\in\R\times(0,1)|\ c(t+k^-)\in\rho_{(t,h)} \text{ for some } k^-\in(0,h)\};
\]
\[
C^+_3:=\{(t,h)\in\R\times(0,1)|\ c(t+k^+)\in\rho_{(t,h)} \text{ for some } k^+\in(h,1)\}.
\]
 
Note that if the two sets $C_3^-$ and $C_3^+$ intersect, then the inscribed polygon defined by the points
$c(t),c(t+k^-),c(t+h),c(t+k^+)$ has total curvature $4\pi$. Hence the total curvature of $\Ga$ would be bounded below by $4\pi$.
 Moreover, an approximation argument shows that it is even enough to show  
\begin{equation*}
\tag{$\star$}\bar C_3^-\cap\bar C_3^+\neq\emptyset.
\end{equation*}
To show that this intersection is indeed nonempty if $\Ga$ is a nontrivial knot,
we need two topological lemmas. The first is taken from \cite{LWint}. 
 

\blem\label{lem_sep}
Let $Q=[0,1]^2$ be the unit square and let $K\subset Q$ be a compact subset.
If $K$ does not separate two points $x,y\in Q$, then $K$ contains a minimal compact subset $K'\subset K$
which still separates $x$ and $y$. Moreover, $K'$ is connected.
\elem

\proof
The existence of $K'$ follows from Zorn's lemma. If $K'$
could be written as a nontrivial disjoint union of compact sets, $K'=K_1\cup K_2$,
then either $K_1$ or $K_2$ would have to separate $x$ and $y$. This would contradict minimality.
Indeed, since $H_1(Q)=0$, the Mayer-Vietoris sequence implies injectivity of
\[H_0(Q \setminus K')\to H_0(Q \setminus K_1)\oplus H_0(Q \setminus K_2).\]
Hence if neither $K_1$ nor $K_2$ separates $x$ and $y$,
then $x$ and $y$ define the same element in $H_0(Q \setminus K')$
and therefore lie in the same component of $Q \setminus K'$.
\qed

Since $\Ga$ is a polygon, it is not hard to see that there exists $\eps>0$ 
such that 
\[d(\bar C_3^\pm,\R\times \{\frac{1\pm 1}{2}\})\geq\eps.\]
Now Lemma~\ref{lem_sep} implies that if $\bar C_3^-$ and $\bar C_3^+$
are disjoint, then $\bar C_3^-\cup\bar C_3^+$ cannot separate the two ends of $\R\times(0,1)$.
Because our setting is periodic, we even find a periodic and piecewise linear path $c:[0,1]\to(\R\times(0,1))\setminus (\bar C_3^-\cup\bar C_3^+)$.

An argument due to Pannwitz shows that this can only happen if $\Ga$ is the unknot.
In order to present the argument, we prepare with the following auxiliary statement which allows to recognize the unknot.

\blem\label{lem_unknot}
Let $\Ga\subset \R^3$ be a polygonal knot. If there exists a continuous disc $u:D\to\R^3$
which restricts to a degree 1 map $\D D\to\Ga$ on the boundary and such that $u^{-1}(\Ga)\subset\D D$,
then $\Ga$ is unknotted.
\elem

\proof
The idea is to use Dehn's Lemma and therefore we need to manipulate $u$ in order to make it piecewise linear and to insure that it restricts to an embedding on 
$\D D$.

We set $\ga=u|_{\D D}$.
We may assume that $u$ is piecewise linear on the complement of a very small neighborhood $U$ of $\Ga$. 
All further neighborhoods of $\Ga$ will be assumed to contain $U$.
There exists a small regular neighborhood $N_1$ of $\Ga$ such that $u$ has a piecewise linear subdisc $u_1$
with $\D u_1=\ga_1\subset\D N_1$ and $\ga_1$ is  homotop to $\ga$ in $\bar N_1$. Since $\ga$ has degree 1,
$\ga_1$ represents an element $(1,m)$ in $\pi_1(\D N_1)$. To get rid of boundary intersections, we choose a second smaller regular neighborhood
$N_2$ of $\Ga$ and a piecewise linear annulus $A$ in $\bar N_2$ with $\D A=\Ga\cup\ga_2$.
By construction, $\ga_1$ and $\ga_2$ are (piecewise linear) homotopic in $\bar N_1\setminus N_2$.
Now we have found a piecewise linear disc filling $\Ga$ without singularities on $\Ga$.
By Dehn's lemma, $\Ga$ bounds an embedded (piecewise linear) disc. Hence $\Ga$ is unknotted.
\qed

\blem[Pannwitz]\label{lem_pannwitz}
If there exists a periodic and piecewise linear curve $c:[0,1]\to\R\times(0,1)$, disjoint from $C_3$, then
$\Ga$ is the unknot.
\elem

\proof
By assumption, the intersection of the ray $\rho_{c(t)}$ with $\Ga$ is contained in the chord $f(c(t))$.
Hence, the continuous family of rays $\rho_{c(t)}$ provide a filling of $\Ga$ by
a disc $u:D\to \R^3$ such that $u^{-1}(\Ga)\subset\D D$. The claim follows from Lemma~\ref{lem_unknot}.
\qed

\proof[Proof of Fary-Milnor]
By our discussion above, it only remains to show property $(\star)$; namely that
the closures of the sets $C_3^-$ and $C_3^+$ intersect.  
However, as mentioned above, if this fails, then Lemma~\ref{lem_sep} implies that 
there exists a periodic and piecewise linear curve $c:[0,1]\to\R\times(0,1)$ disjoint from $C_3$.
By Lemma~\ref{lem_pannwitz}, $\Ga$ has to be the unknot. Contradiction.
\qed

\bibliographystyle{alpha}
\bibliography{fm_sur}

%\Addresses

\end{document}















\grid
\grid


