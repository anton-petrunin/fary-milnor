\section{Cantarella--Kuperberg--Kusner--Sullivan}\label{sec:2nd-hull}

The following proof is due to Jason Cantarella, Greg Kuperberg, Robert Kusner, and John Sullivan \cite{CKKS};
this is the only proof in our collection that use the knot theory a bit beyond the definitions.

Choose a polygonal curve $\alpha=p_1\dots p_n$.
Suppose a plane $\Pi$ is in general position; that is, it does not contain vertices of $\alpha$.
Let us denote by $\cross_\alpha\Pi$ the number of intersections of $\Pi$ and $\alpha$.
Extend the function $\Pi\mapsto \cross_\alpha(\Pi)$ to the minimal upper semicontinuous function defined for all planes;
in other words, $\cross_\alpha(\Pi)$ is the maximal integer $k$ such that there is a plane $\Pi'$ arbitrary close to $\Pi$ that intersects $\alpha$ at $k$ points.
The number $\cross_\alpha(\Pi)$ will be called the \emph{crossing number} of $\Pi$.
Note that for any plane $\Pi$, the number $\cross_\alpha(\Pi)$ is even, and it cannot exceed $n$.

It is easy to see that the \emph{convex hull} $h_1(\alpha)$ of $\alpha$ can be defined the following way:
\textit{$x\in h_1(\alpha)$ if $\cross_\alpha(\Pi)\ge2$ for any plane $\Pi\ni x$}.
This observation suggests the following definition of the \emph{second hull}:
\textit{$x\in h_2(\alpha)$ if $\cross_\alpha(\Pi)\ge4$ for any plane $\Pi\ni x$}.

\begin{thm}{Theorem}\label{thm:2nd-hull}
The second hull оf any nontrivial knot $\alpha$ has a nonempty interior.
\end{thm}

To show that this theorem implies our main theorem, we will apply the spherical Crofton formula:
\textit{for any spherical polygonal curve $\gamma$ we have 
\[\length \gamma=\pi\cdot \overline n,\leqno({*})\]
where $\overline n$ denotes the average number of intersections of $\gamma$ with equators.}
To prove this formula, check it for an arc on an equator and sum it up for all edges of $\gamma$.

\parit{Proof of \ref{thm:fary-milnor} modulo \ref{thm:2nd-hull}.}
Choose a point $o\in h_2(\alpha)$; we can assume that $o\notin\alpha$.
Consider the radial projection $\alpha^*$ of $\alpha$ to the sphere centered at $o$;
observe that 
\[\length\alpha^*=\Psi_o(\alpha),\]
where $\Psi_o(\alpha)$ denotes the angular length of $\alpha$ with respect to $o$; see Section~\ref{sec:fary}.

By \ref{prop:angular-length}, it is sufficient to show that 
\[\length\alpha^*\ge 4\cdot\pi.\leqno({*}{*})\]
Since $o$ is in the second hull, $\alpha^*$ crosses every equator in general position at least 4 times.
It follows that the average number of crossings is at least~4.
Applying $({*})$, we get $({*}{*})$.
\qeds

Suppose that a plane $\Pi$ divides a knot $\alpha$ into two arcs, one on each side; in particular, $\Pi$ intersects $\alpha$ at two points, say $p$ and $q$.
Then we can create two knots $\alpha_1$ and $\alpha_2$ by joining ends of the two arcs by the line segment $[pq]$.
In this case, we say that $\alpha$ is a \emph{connected sum} of $\alpha_1$ and $\alpha_2$.

\begin{thm}{Claim}\label{clm:connected-sum}
Suppose that a knot $\alpha$ is a connected sum of knots $\alpha_1$ and $\alpha_2$.
If at least one of the knots $\alpha_1$ or $\alpha_2$ is nontrivial, then so is $\alpha$.
\end{thm}

This claim has an amusing proof via the so-called \emph{infinite swindle} \cite{mazur}.

Suppose $\beta$ is a polygonal curve \emph{inscribed} in $\alpha$;
that is, the vertices of $\beta$ lie on $\alpha$ and they appear in the same cyclic order on $\alpha$ and $\beta$.
If a plane in general position intersects an edge of $\beta$, then it intersects corresponding arc of $\alpha$.
Therefore we get the following:

\begin{thm}{Observation}
If a polygonal line $\beta$ is inscribed in $\alpha$.
Then $h_2(\beta)\subset h_2(\alpha)$.
\end{thm}

\parit{Proof of \ref{thm:2nd-hull}.}
Assume the contrary; let $\alpha$ be a nontrivial polygonal knot with the least number of vertices, say $n$, such that $h_2(\alpha)$ has empty interior.
It is easy to see that $n\ge 6$;
in other words, any space 5-gon is a trivial knot.

Suppose $\Pi$ is a plane in general position that divides $\alpha$ in two arcs, so it defines a decomposition of $\alpha$ into a connected sum of two knots $\alpha_1$ and $\alpha_2$.
Note that one of these knots, say $\alpha_1$, is trivial and therefore the other, respectively $\alpha_2$, is isotopic to $\alpha$.
Indeed, by the observation, $h_2(\alpha_1)$ and $h_2(\alpha_2)$ have empty interior.
So it is sufficient to show that $\alpha_1$ or $\alpha_2$ has fewer vertices than $\alpha$.
The latter statement easily follows since the total number of vertices in $\alpha_1$ and $\alpha_2$ is $n+4$ and $n\ge 6$.

If $\alpha_1$, $\alpha_2$, and $\Pi$ as above, then the open half-space $H$ bounded by $\Pi$ and containing $\alpha_2$ will be called \emph{essential};
plus any open half-space containing whole $\alpha$ will be called essential as well.
A rather straightforward application of \ref{clm:connected-sum} implies that if $H'$ is another essential half-space for $\alpha$, then it is also essential for $\alpha_2$.
It follows that intersection, say $W$, of all essential half-spaces has a nonempty interior --- roughly speaking, it has to contain the region where knotting of $\alpha$ takes place.

Finally observe that if $\cross_\alpha\Pi=2$ for a plane $\Pi$ in general position, then $\Pi$ bounds an essential half-space. 
It follows that if a plane in general position intersects $W$, then it has the crossing number at least 4,
so $h_2(\alpha)\supset W$ --- hence the result.
\qeds

