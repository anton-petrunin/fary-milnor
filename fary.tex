\section{F\'ary's proof}

Let us give a sketch of another proof, based on the original idea of Istv\'an F\'ary.

\begin{wrapfigure}{r}{30 mm}
\vskip-0mm
\centering
\includegraphics{mppics/pic-13}
\vskip0mm
\end{wrapfigure}

\parit{F\'ary's proof of \ref{thm:fary-milnor}.}
Consider a projection of the knot to a plane in general position.
That is, we assume that the self-intersections of the projection are at most double and the projection of each edge is not degenerate.
The obtained closed polygonal line $\beta=p_1p_2\dots p_n$ divides the plane into domains, one of which is unbounded, denote it by $U$, and the others are bounded.

First note that all domains can be colored in a chessboard order;
that is, they can be colored in black and white in such a way that domains with common borderline get different colors.
If the unbounded domain is colored in white and every other domain is colored in black then one can untie the knot by flipping these domains one by one.

\begin{thm}{Exercise}
Give a formal proof of the last statement; that is, show that if the only undbounded domain is white then $\beta$ is isotopic to a triangle. 
\end{thm}

\begin{wrapfigure}{r}{30 mm}
\vskip-4mm
\centering
\includegraphics{mppics/pic-14}
\vskip0mm
\end{wrapfigure}

Therefore among the bounded domains there is a white domain, denote it by $D$.
The domain $D$ cannot adjoin %the word adjoin is used for countries, and I guess it can be used in the plane.
$U$, since they have the same color.
Fix a point $o$ in this domain.

For each $i$, set 
\begin{align*}
\phi_i&=\pi-\measuredangle p_{i-1}p_ip_{i+1},
\\
\psi_i&=\measuredangle p_{i-1} o p_{i},
\\
\theta_i&=\measuredangle o p_i p_{i+1}.
\end{align*}
Here indexes are taken modulo $n$; in particular, $p_{n}=p_0$.


Note that $\phi_i$ is the external angle at $p_i$;
therefore 
\[\tc\beta= \phi_1+\dots+\phi_n\]

Direct calculations show that 
\[\phi_i\ge \psi_i+\theta_{i-1}-\theta_i.\]
In the two pictures below, $\phi_i$ is the solid angle and 
the angles $\psi_i$, $\theta_{i-1}$ and $\theta_i$
are just as drawn.
We have equality on the first picture and strict inequality on the second picture.

\begin{figure}[h]
\vskip-0mm
\centering
\includegraphics{mppics/pic-15}
\vskip0mm
\end{figure}

It follows that 
\[\phi_1+\dots+\phi_n\ge \psi_1+\dots+\psi_n.\]

The last sum 
is the total angle at  which $\beta$ is seen from $o$ counted with multiplicity. 
The boundary of $D$ contributes at least $2\cdot\pi$ to this sum and the boundary of $U$ contributes with other $2\cdot\pi$;
since their boundaries do not overlap we get 
\[\psi_1+\dots+\psi_n\ge 4\cdot\pi,\]
hence the result.

This is true for the projection of the knot to any plane in general position.
The remaining planes contribute nothing to the average value.
Therefore by Proposition~\ref{prop:tc-crofton}, the total curvature of the original knot is at least $4\cdot\pi$.
\qeds



\begin{thm}{Exercise}
Construct a closed smooth simple curve with total curvature arbitrarily close to $2\cdot\pi$ such that its projection to any plane has at least $10$ self-intersections.   
\end{thm}



\section{Proof of Alexander and Bishop}

Here we sketch a proof of the F\'ary--Milnor theorem given by of Stephanie Alexander and Richard Bishop in \cite{alexander-bishop}.

The proof is elementary, but not simple 
(elementary does not mean simple, it means only that it does not use much theory).
It is based on the following two facts that we are already familiar with:
\begin{itemize}
\item If a closed polygonal line $\beta'$ is inscribed in a closed polygonal line $\beta$ then 
 \[\tc\beta'\le \tc\beta.\]
\item The total curvature of a doubly covered
bigon is $4\cdot\pi$; that is,
\[\tc\beta=4\cdot\pi\]
if $\beta=pqpq$ for two distinct points $p$ and $q$.
Similarly if a quadrilateral is sufficiently close to a doubly covered
bigon, then its total curvature is close to $4\cdot\pi$.
\end{itemize}


\parit{Proof.}
Let $\beta=p_1\dots p_n$ be a closed polygonal line that is not a trivial knot;
that is, one can not get a triangle from $\beta$ by applying a sequence of triangular isotopies defined in the previous section.

We proceed by induction on the number $n \ge 3$.
In the base case $n=3$ the polygonal line $\beta$ is a triangle.
Therefore, by definition, $\beta$ is a trivial knot --- nothing to show.

Consider the smallest $n$ for which the statement fails;
that is, there is a closed simple polygonal line $\beta\z=p_1\dots p_n$ that is not a trivial knot and such that
\[\tc\beta<4\cdot\pi.
\eqlbl{eq:<4pi}\]
We use the indexes modulo $n$; that is, $p_0=p_n$, $p_1=p_{n+1}$ and so on.
Without loss of generality, we may assume that $\beta$ is in general position; 
that is, no four vertexes of $\beta$ lie on one plane. 

Set $\beta_0=\beta$.
If the solid triangle $\solidtriangle p_{0}p_1p_{2}$ intersects $\beta_0$ only in the two adjacent edges,
then applying the corresponding triangular isotopy, we get a knot $\beta'_0$ with $n-1$ edges that is inscribed in $\beta_0$. Therefore
\[\tc\beta_0\ge \tc\beta_0'.\]
On the other hand, by the induction hypothesis 
\[\tc\beta_0'\ge 4\cdot\pi , \]
which contradicts \ref{eq:<4pi}.

Choose the first point $w'_1$ on the edge $[p_1p_2]$ so that the line segment $[p_0w'_1]$ 
intersects $\beta_0$.
Denote a point of intersection by $y_1$.

\begin{wrapfigure}{r}{25 mm}
\vskip-0mm
\centering
\includegraphics{mppics/pic-17}
\vskip0mm
\end{wrapfigure}

Choose a point $w_1$ on $[p_1p_2]$ a bit before $w'_1$
(below we explain how close).
Denote by $x_1$ the point on $[p_0w_1]$ that minimizes the distance to $y_1$.
This way we get a closed polygonal line 
$\beta_1\z=w_1p_2\dots p_n$ with two marked points $x_1$ and $y_1$.
Denote by $m_1$ the number of edges in the arc $x_1w_1\dots y_1$ of $\beta_1$.

By Exercise~\ref{ex:triangle-isotopy}, $\beta_1$ is isotopic to $\beta_0$;
in particular $\beta_1$ is a nontrivial knot.

Now let us repeat the procedure for the adjacent edges $[w_1p_2]$ and $[p_2p_3]$ of $\beta_1$.
If the solid triangle $\solidtriangle w_1p_2p_3$ intersects $\beta_1$ only at these two adjacent edges, then we get a contradiction with the induction hypothesis the same way as before.
Otherwise we get a new knot $\beta_2=w_1w_2p_3\dots p_n$ with two marked points $x_2$ and $y_2$.
Denote by $m_2$ the number of edges in the broken line $x_2w_2\dots y_2$.

Note that the points $x_1,x_2,y_1,y_2$ can not appear on $\beta_2$ in the same cyclic order;
otherwise the broken line $x_1x_2y_1y_2$ can be made to be arbitrary close to a doubly covered bigon which again contradicts~\ref{eq:<4pi}.%
\footnote{More precisely, the choice of $w_1$ has to be made so that the distance $|x_1-y_1|$ would be much less that all the distances between $y_1$ and any point $z\in\beta\cap \solidtriangle p_1p_2p_3$, so we have
\[\measuredangle y_1zx_1<\tfrac\varepsilon{10},\]
where $\varepsilon=4\cdot\pi-\tc\beta$.
In this case, since $y_2\in \beta\cap \solidtriangle p_1p_2p_3$ and $x_2$ can be taken arbitrary close to $y_2$, we have
\[\tc x_1x_2y_1y_2 > 4\cdot\pi -\varepsilon= \tc\beta\]
which can not happen since $x_1x_2y_1y_2$ is inscribed in $\beta$.}

Therefore we can assume that the arc $x_2w_2\dots y_2$ lies inside the arc $x_1w_1\dots y_1$ in $\beta_2$
and therefore $m_1>m_2$.

Continuing this procedure we get a sequence of polygonal lines $\beta_i\z=w_1\dots w_i p_{i+1}p_n$ with marked points $x_i$ and $y_i$ such that the number of edges $m_i$ from $x_i$ to $y_i$ decreases as $i$ increases.
Clearly $m_i>1$ for any $i$ and $m_1<n$.
Therefore it requires less than $n$ steps to get a contradiction with the induction hypothesis.
\qeds

\begin{wrapfigure}{r}{30 mm}
\vskip-0mm
\centering
\includegraphics{mppics/pic-20}
\vskip0mm
\end{wrapfigure}

\begin{thm}{Exercise}
Suppose that a closed curve $\alpha$ crosses a line at four points $a$, $b$, $c$ and $d$.
Assume that the points $a$, $b$, $c$ and $d$ appear on the line in that order and they appear on the curve $\alpha$ in the order $a$, $c$, $b$, $d$.
Show that 
\[\tc\alpha\ge 4\cdot\pi.\]
\end{thm}

A line crossing a knot at four points as in the exercise is called \emph{alternating quadrisecants}.
It turns out that any nontrivial knot admits an alternating quadrisecants \cite{denne};
it provides yet another proof of the F\'ary--Milnor theorem.


\begin{thm}{Advanced exercise}
Show that given any real number $\Phi$ there is a knot $\beta$ such that any knot isotopic to $\beta$ has total curvature at least~$\Phi$.   
\end{thm}

\parit{Hint:} Use that there are knots with arbitrary large \emph{bridge number}, see for example \cite{schultens} and the references therein.
