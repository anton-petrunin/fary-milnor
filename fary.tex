\section{F\'ary}\label{sec:fary}

In this section, we sketch the solution of Istv\'an F\'ary \cite{fary} which was published before Milnor's proof.

We start with Crofton-type formulas for total curvature.
Given a curve $\alpha$ in $\mathbb{R}^3$ and a unit vector $u$, denote by $\alpha_{u^\perp}$ 
and $\alpha_u$ the projections of $\alpha$ to the plane perpendicular to $u$ and the line parallel to $u$ respectively.
Given a function $f\:\mathbb{S}^2\to\mathbb{R}$, let us denote its average value by $\overline{f(u)}$.

\begin{thm}{Crofton-type formula}\label{prop:tc-crofton}
Let $\alpha$ be a polygonal curve in $\mathbb{R}^3$.
Then
\begin{align*}
\tc\alpha
&=\overline{\tc{\alpha_{u^\perp}}}
=\overline{\tc{\alpha_u}}.
\end{align*}
\end{thm}


\parit{Proof.}
Observe that it is sufficient to check the identities for polygonal curves made of two edges
and in this case, it boils down to very straightforward calculations.
\qeds

The original version of Milnor's proof 
used the identity $\tc\alpha=\overline{\tc{\alpha_u}}$;
in Fenner's version of his proof it was hidden under the carpet.

F\'ary's proof is based on the identity $\tc\alpha=\overline{\tc{\alpha_{u^\perp}}}$ and the following inequality for total curvature.
Suppose $\alpha=p_1\dots p_n$ is a simple closed polygonal curve in $\mathbb{R}^3$ and $o\notin\alpha$.
Let us define the \emph{angular length} of $\alpha$ with respect to $o$ as the sum
\[\Psi_o(\alpha)=\measuredangle p_{1} o p_{2}+\dots+\measuredangle p_{n-1} o p_{n}+\measuredangle p_{n} o p_{1}.\]

\begin{thm}{Proposition}\label{prop:angular-length}
For any closed simple polygonal curve and any $o\notin\alpha$, we have 
\[\Psi_o(\alpha)\le \tc{\alpha}.\]
\end{thm}

\parit{Proof.}
Let $\alpha=p_1\dots p_n$; for each $i$, set 
\begin{align*}
\phi_i&=\pi-\measuredangle p_{i-1}p_ip_{i+1},
&
\psi_i&=\measuredangle p_{i-1} o p_{i},
&
\theta_i&=\measuredangle o p_i p_{i+1}.
\end{align*}
Here we assume that indexes are taken modulo $n$; in particular, $p_{n}=p_0$.

\begin{wrapfigure}{r}{45 mm}
\vskip-0mm
\centering
\includegraphics{mppics/pic-15}
\vskip0mm
\end{wrapfigure}

Note that $\phi_i$ is the external angle at $p_i$;
therefore 
\[\tc\alpha= \phi_1+\dots+\phi_n\]

The directions of $p_i-p_{i-1}$, $o-p_i$, and $p_{i+1}-p_i$ make angles 
$\psi_i+\theta_{i-1}$, $\theta_i$, and $\phi_i$ to each other.
Applying the triangle inequality for these angles, we get
\[\phi_i\ge \psi_i+\theta_{i-1}-\theta_i.\]
Summing up, we get
\[\phi_1+\dots+\phi_n\ge \psi_1+\dots+\psi_n,\]
and the result follows.
\qeds


\begin{wrapfigure}{r}{44 mm}
\vskip-0mm
\centering
\includegraphics{mppics/pic-13}
\vskip0mm
\end{wrapfigure}

\parit{F\'ary's proof.}
Consider a projection of the knot to a plane in \emph{general position}
(this time it means that the self-intersections of the projection are at most double and the projection of each edge is not degenerate).
The obtained closed polygonal curve $\alpha_{u^\perp}=p_1p_2\dots p_n$ divides the plane into domains, one of which is unbounded, denote it by $U$, and the others are bounded.

First, note that all domains can be colored in a chessboard order;
that is, they can be colored in black and white in such a way that domains with common borderline get different colors.
If the unbounded domain is colored in white and every other domain is colored in black, then one can untie the knot by flipping these domains one by one.%
\footnote{It is instructive to give a formal proof of the last statement; that is, \textit{show that if there is only one white region, then $\alpha$ is trivial}.}


\begin{wrapfigure}{r}{44 mm}
\vskip-4mm
\centering
\includegraphics{mppics/pic-14}
\vskip0mm
\end{wrapfigure}

Therefore, among the bounded domains there is a white domain, denote it by $D$.
The domain $D$ cannot adjoin %the word adjoin is used for countries, and I guess it can be used in the plane. I agree.
$U$, since they have the same color.
Fix a point $o$ in this domain.

Since any ray from $o$ crosses $\alpha_{u^\perp}$ twice, we get $\Psi_o(\alpha_{u^\perp})\ge 4\cdot\pi$;
that is, the angular length of $\alpha_{u^\perp}$ with respect to $o$ is at least $4\cdot\pi$. 
By \ref{prop:angular-length}, we have 
\[\tc{\alpha_{u^\perp}}\ge4\cdot\pi.\]
This is true for any $u$ in general position.
The remaining directions contribute nothing to the average value.
It remains to apply the Crofton-type formula $\tc{\alpha}=\overline{\tc{\alpha_{u^\perp}}}$.
\qeds
