\section{Pannwitz--Hopf--Schmitz--Denne}\label{sec:quadrisecant}

Fáry's paper \cite{fary} is ended with the following note:
\textit{I just received a letter from Mr. Borsuk, which says that Theorem 3} (our main theorem) \textit{has been proved independently by Mr. H. Hopf.
It uses the theorem of Miss Pannwitz, which ensures that for any knot there is a line that crosses it at least in four
points.}
%Je viens de recevoir une lettre de M. Borsuk, qui m'apprend que le théorème 3 a été démontré indépendamment par M. H. Hopf. Il utilise le théorème de Mlle Pannwitz, qui assure que pour tout nœud on peut trouver une droite le coupant au moins en quatre points.
This proof is the subject of this section.

{

\begin{wrapfigure}{r}{22 mm}
\vskip-0mm
\centering
\includegraphics{mppics/pic-41}
\caption*{Alternated quadrisecant.}
\vskip0mm
\end{wrapfigure}

The main step in the proof is the existence of the so-called \emph{alternated quadrisecant} of a knot $\alpha$; that is, there is a line $\ell$ that shares with $\alpha$ four points $a$, $b$, $c$, and $d$
that appear on $\ell$ in the same order and have cyclic order $a$, $c$, $b$, $d$ on $\alpha$.

The question about the existence of a quadrisecant was asked by Otto Toep\-litz and answered by Erika Pannwitz for tame knots in \emph{general position} \cite{pannwitz}.
Erika Pannwitz only proved the existence of a quadrisecant, but the existence of alternated quadrisecant can be extracted from her proof.
Elizabeth Denne had generalized this result to all knots \cite{denne, denne-survey}.
But for us tame knots in general position will be sufficient;
namely, we need the following:

}

\begin{thm}{Proposition}\label{prop:quadrisecant}
Any nontrivial knot $\alpha$ in general position admits an \emph{alternated quadrisecant}.
\end{thm}

The precise meaning of general position will be clear from the proof;
what is important is that any knot is arbitrarily close to a knot in general position.

\parit{Proof of the main theorem modulo \ref{prop:quadrisecant}.}
Let $\alpha=p_1\dots p_n$ be a knot in general position.
Suppose $a$, $b$, $c$, and $d$ be the points as in the definition of an alternated quadrisecant.
Then $acbd$ is an inscribed quadrangle with all external angles equal to $\pi$.
Therefore $\tc{abcd}=4\cdot\pi$.

Since the total curvature of an inscribed polygonal curve cannot be larger than the total curvature of the original curve, we get that $\tc{\alpha}\z\ge 4\cdot\pi$.

Finally, for any knot $\beta=q_1\dots q_n$ there is an arbitrarily close knot $\alpha\z=p_1\dots p_n$ in general position;
in particular, for any $\eps>0$ we can assume that $\tc{\beta}>\tc\alpha-\eps$.
It follows that $\tc{\beta}>4\cdot\pi-\eps$ for any positive $\eps$ --- hence the result.
\qeds

In the proof of the proposition, we will use the following characterization of the trivial knot.

\begin{thm}{Lemma}
A knot $\alpha$ is trivial if there exists a piecewise linear map $F$ from the disc $\DD$ to $\RR^3$ such that the  restriction of $F$ to the boundary $\partial\DD=\mathbb{S}^1$ is a degree-one map to $\alpha$ and $F$ does not map interior points of $\DD$ to $\alpha$.
\end{thm}

This statement can be deduced from  the \textit{loop theorem} --- a heavy weapon of 3-dimensional topology.
For those who are familiar with the loop theorem, it would be an exercise; otherwise, we suggest taking it for granted.
In the following proof, we follow closely the presentation of Carsten Schmitz \cite{schmitz} and the original argument of Erika Pannwitz.

\parit{Proof of the proposition.}
We may assume that $\alpha$ comes with a 1-periodic piecewise linear parametrization by $\RR$;
so the space of oriented chords of $\alpha$ can be identified with the open cylinder $\mathbb{S}^1\times (0,1)$, where $\mathbb{S}^1=\RR/\ZZ$.
Namely, we assume that a pair $(x,y)\in \mathbb{S}^1\times (0,1)$ corresponds to the oriented chord with the ends at $\alpha(x)$ and $\alpha(x+y)$.

\begin{wrapfigure}[7]{r}{42 mm}
\vskip-6mm
\centering
\includegraphics{mppics/pic-43}
\caption*{Ray $r(x,y)$.}
\vskip0mm
\end{wrapfigure}

Choose a pair $(x,y)\in \mathbb{S}^1\times (0,1)$.
Let us denote by $r(x,y)$ the ray that starts at $\alpha(x)$ and goes in the direction opposite to $\alpha(x+y)$.
We write $(x,y)\in C_3$ if $r(x,y)$ crosses $\alpha$ at another point.

The points $\alpha(x)$ and $\alpha(x+y)$ divide $\alpha$ into two open arcs $\alpha|_{(x,x+y)}$ and $\alpha|_{(x+y,x+1)}$.
If $(x,y)\z\in C_3$ and $r(x,y)$ crosses the second arc, then we write $(x,y)\in C_3^+$;
if it crosses the first arc, then $(x,y)\z\in C_3^-$.
Note that $C_3^+\cup C_3^-\z=C_3$.

Observe that if $C_3^+$ and $C_3^-$ intersect, then the proposition follows.
The set $C_3^\pm$ is not closed;
denote by $\bar C_3^\pm$ its closure.
Suppose 
\[\bar C_3^+\cap \bar C_3^-=\emptyset.\]
Note that in this case there is a large $n$ such that any $\tfrac1n\times\tfrac1n$-square in $\mathbb{S}^1\times (0,1)$ does not intersect both $\bar C_3^+$ and $\bar C_3^-$.
Let us cut $\mathbb{S}^1\times (0,1)$ into $n^2$ such squares; each square is a closed subset of $\mathbb{S}^1\times (0,1)$.
Color the union of squares that intersect $\bar C_3^+$.
Note that every square in the lowest row is colored and we did not color squares in the upper row.
Further, the boundary of the colored set contains a simple curve $t\mapsto(x(t),y(t))$ that cuts  the cylinder $\mathbb{S}^1\times (0,1)$ into two cylinders.

\begin{wrapfigure}{r}{24 mm}
\vskip-0mm
\centering
\includegraphics{mppics/pic-46}
\caption*{Colored set.}
\vskip0mm
\end{wrapfigure}

Note that $(x(t),y(t))\notin \bar C_3^\pm$ for any $t\in \mathbb{S}^1$ and the curve $t\mapsto(x(t),y(t))$ runs along coordinate lines.
Consider the one-parameter family of line segments in $r(x(t),y(t))$ that start at $\alpha(x(t))$ and end on the surface of a large cube that contains $\alpha$ in its interior.
This way we obtain a piecewise linear annulus that connects the curve $t\mapsto \alpha(x(t))$ to a curve on the surface of the cube.
The latter curve can be contracted by a piecewise linear disc in the surface of the cube.
It might have self-intersection, but it cannot contains points of $\alpha$.

Observe that $t\mapsto \alpha(x(t))$ defines a degree-one map $\mathbb{S}^1\z\to\alpha$.
Applying the lemma, we get the result.

It remains to show that if $\alpha=p_1\dots p_n$ is in general position, then
\[C_3^+\cap C_3^-=\emptyset\quad\text{implies that}\quad
\bar C_3^+\cap \bar C_3^-=\emptyset.\]
Assume $C_3^+\cap C_3^-=\emptyset$ and $(x,y)\in \bar C_3^+\cap \bar C_3^-$.
Denote by $\ell$ the line containing $\alpha(x)$ and $\alpha(x+y)$.
Checking the following statements is straightforward, but requires patience:
\begin{itemize}
 \item $\alpha(x)$ is a vertex, so $\alpha(x)=p_i$ for some $i$ and $\ell$ contains $p_i$;
 \item $\ell$ lies in the plane spanned by $p_{i-1}$, $p_i$, and $p_{i+1}$; 
 \item $\ell$ does not contain edge $[p_{i-1}, p_i]$, nor $[p_{i}, p_{i+1}]$;
 \item $\ell$ has at least 3 points of intersection with $\alpha$.
\end{itemize}
Finally, if $\alpha$ is in general position,
the line $\ell$ with the described properties does not exist.
\qeds





